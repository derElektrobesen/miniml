\documentclass[a4paper,12pt]{article}

% polyglossia should go first!
\usepackage{polyglossia} % multi-language support
\setmainlanguage{russian}
\setotherlanguage{english}

\usepackage{amsmath} % math symbols, new environments and stuff
\usepackage{unicode-math} % for changing math font and unicode symbols
\usepackage[style=english]{csquotes} % fancy quoting
\usepackage{microtype} % for better font rendering
\usepackage{hyperref} % for refs and URLs
\usepackage{graphicx} % for images (and title page)
\usepackage{geometry} % for margins in title page
\usepackage{tabu} % for tabulars (and title page)
\usepackage[backend=bibtex, style=numeric-comp, sorting=none]{biblatex} % for bibliography
\usepackage{placeins} % for float barriers
\usepackage{titlesec} % for section break hooks
\usepackage{listings} % for listings 
\usepackage{upquote} % for good-looking quotes in source code (used for custom languages)
\usepackage{xcolor} % colors!
\usepackage{enumitem} % for unboxed description labels (long ones)
\usepackage{caption}

\lstloadlanguages{C,haskell,bash,java}

\lstset{
    frame=none,
    xleftmargin=2pt,
    stepnumber=1,
    numbers=left,
    numbersep=5pt,
    numberstyle=\ttfamily\tiny\color[gray]{0.3},
    belowcaptionskip=\bigskipamount,
    captionpos=b,
    escapeinside={*'}{'*},
    language=haskell,
    tabsize=2,
    emphstyle={\bf},
    commentstyle=\it,
    stringstyle=\mdseries\rmfamily,
    showspaces=false,
    keywordstyle=\bfseries\rmfamily,
    columns=flexible,
    basicstyle=\small\sffamily,
    showstringspaces=false,
    morecomment=[l]\%,
    breaklines=true
}
\renewcommand\lstlistingname{Листинг}

\defaultfontfeatures{Mapping=tex-text} % for converting "--" and "---"
\setmainfont{CMU Serif}
\setsansfont{CMU Sans Serif}
\setmonofont{CMU Typewriter Text}
\setmathfont{XITS Math}
\MakeOuterQuote{"} % enable auto-quotation

% new page and barrier after section, also phantom section after clearpage for
% hyperref to get right page.
% clearpage also outputs all active floats:
\newcommand{\sectionbreak}{\clearpage\phantomsection}
\newcommand{\subsectionbreak}{\FloatBarrier}
\newcommand\numberthis{\addtocounter{equation}{1}\tag{\theequation}}
\renewcommand{\thesection}{\arabic{section}} % no chapters
\numberwithin{equation}{section}
%\usetikzlibrary{shapes,arrows,trees}

% 20 - 25 стр достаточно

\bibliography{reference_list}

\begin{document}

\tableofcontents

\section{Введение}
В настоящей работе в рамках курса <<конструирование компиляторов>>
реализуется фронтенд компилятора MiniML. В записке приводятся
использованная грамматика языка, описания лексического,
синтаксического, семантического анализаторов. Кратко описан поиск
лексических, синтаксических и семантических ошибок, таких как
несогласованность типов. Результатом курсовой работы является ПО,
которое при удачном лексическом, синтаксическом и семантическом
разборах создаёт абстрактное синтаксическое дерево, соответствующее
тексту исходной программы.

\section{Выбор платформы} 
Для реализации фронтенда компилятора было решено использова язык
Haskell. Такое решение было принято из-за следующих преимуществ языка:
\begin{enumerate}
\item статическая, сильная типизация;
\item автоматический вывод типов, основанный на алгоритме Хиндли -- Милнера;
\item ленивые вычисления;
\item удобство написания парсеров.
\end{enumerate}
Был использован компилятор GHC 7.10.1 \cite{ghc}, система сборки и управления
пакетами и библиотеками языка Haskell Cabal 1.22.4.0 \cite{cabal}.


Для написания лексического и синтаксического анализаторов использовалась
библиотека parsec \cite{parsec}. Для сериализации AST в JSON-формат
использовалась библиотека aeson \cite{aeson}.

\section{Описание языка}
Язык имеет следующие конструкции:
\begin{enumerate}
\item Беззнаковые целые числа и определённые над ними операции: +, - и *.
\item Логический тип (true, false), условные выражения и сравнение
  целых чисел (только = и <).
\item Рекурсивные функции и применение функций. Выражение 
  fun f (x : t) : s is e обозначает функцию типа t -> s, в которой
  инструкциям в e доступен x.
\item Высокоуровневые объявления: let x = e. Локальных объявлений нет.
\end{enumerate}

\section{Лексический анализ}
\subsection{Входные и выходные типы данных}
На этапе лексического анализа происходит преобразование входной строки
в последовательность токенов. Токеном может быть: идентификатор, оператор,
разезервированное слово, константа типа unsigned int или bool.

\subsection{Обнаруживаемые ошибки}
Единственные ошибки, обнаруживаемые фронтендом компилятора на этапе
лексического анализа, вызываются наличием в исходном коде компилируемой
программы недостимых символов. К примеру, такую ошибку вызовет встреченный
анализатором в любом месте программы символ \%.


Ошибки на этапе лексического анализа являются критическими: при возникновении
ошибки работа фронтенда немедленно завершается выводом сообщения, содержащего
позицию ошибки и подсказку о возможных способах её исправления.

\begin{lstlisting}[ language=haskell
                  , caption=Сообщение о лексической ошибке
                  , label=lst:lexical_error]
MiniML> let asd%dsa = 2
(line 1, column 8):
unexpected "%"
expecting letter or "="MiniML> let asd%dsa = 2
(line 1, column 8):
unexpected "%"
expecting letter or "="
\end{lstlisting}

Пример сообщения об ошибке, вызываемой наличием символа, который не
допускает грамматика языка, в исходном коде программы \ref{lst:lexical_error}.

\subsection{Реализация лексического анализатора}
Лексический анализатор был реализован с помощью библиотеки parsec,
которая позволяет с помощью комбинаторов из простых парсеров собирать
более сложные.


Исходный код лексического анализатора представлен в приложении 1.

\section{Синтаксический анализ}
\subsection{Входные и выходные типы данных}
На этапе синтаксического анализа происходит преобразование
последовательности токенов, сгенерированной лексическим анализатором,
в абстрактное синтаксическое дерево.


Описание выходные типов парсера представлено в листинге \ref{lst:parser}. 
Корнем дерева является команда высшего уровня ToplevelCmd.

\begin{lstlisting}[ language=haskell
                  , caption=Выходные типы данных парсера
                  , label=lst:parser]
-- Variable names
type Name = String

-- Types
data Ty = TInt         -- integers
        | TBool        -- booleans
        | TArrow Ty Ty -- functions

-- Expressions
data Expr = Var Name
          | Int Integer
          | Bool Bool
          | Times Expr Expr
          | Plus Expr Expr
          | Minus Expr Expr
          | Equal Expr Expr
          | Less Expr Expr
          | If Expr Expr Expr
          | Fun Name Name Ty Ty Expr
          | Apply Expr Expr

-- Toplevel commands
data ToplevelCmd = Expr Expr
                 | Def Name Expr
\end{lstlisting}

\subsection{Обнаруживаемые ошибки}
На данном этапе обнаруживаются ошибки несоответствия исходного кода
грамматике языка. Ошибки данного типа также являются критическими,
хотя более сложные парсеры могут восстанавливаться после ошибок.


При возникновении ошибки работа фронтенда немедленно завершается
выводом сообщения об ошибке, содержащего позицию ошибки и подсказку о
возможных способах её устранения.


Примеры сообщений об ошибках представлены в листинге \ref{lst:syntax_errors}.

\begin{lstlisting}[ language=haskell
                  , caption=Сообщения о синтаксической ошибке
                  , label=lst:syntax_errors]
MiniML> let x = (
(line 1, column 10):
unexpected end of input
expecting "(", "fun", natural, identifier, "if", "true" or "false"

MiniML> ((23 + 3) 
(line 1, column 10):
unexpected end of input
expecting ")"

MiniML> let a = + 2
(line 1, column 9):
unexpected "+"
expecting "(", "fun", natural, identifier, "if", "true" or "false"
\end{lstlisting}

\subsection{Реализация парсера}
Парсер был реализован с помощью библиотеки parsec. Реализация находится
в приложении 2.

\section{Семантический анализ}
\subsection{Вывод типов}
Вывод типов позволяет компилятору узнать тип значения выражения, не
указывая его явно. Например, ясно, что выражение e1 + e2 вычисляется
в тип int, а его операнды e1 и e2 также должны быть типа int.


Абстрактное синтаксическое дерево, полученное на выходе синтакического
анализатора проверяет модулем TypeCheck.hs на согласованность типов.


Исходный код модуля представлен в приложении 3.

\subsection{Обнаружение ошибок}
Примеры обнаружение ошибок несогласованности типов представлены
в листинге \ref{lst:type_check_errors}.

\begin{lstlisting}[ language=haskell
                  , caption=Сообщения о несогласованности типов
                  , label=lst:type_check_errors]
MiniML> let a = true
a : bool
MiniML> a + 2
miniml: TE "Var "a" has type bool but is used as if it has type int"

MiniML> if 1 then false else true
miniml: TE "Int 1 has type int but is used as if it has type bool"
\end{lstlisting}

\section{Вывод}


\section{Список литературы}
\printbibliography[heading=none]

\section*{\titleline[r]{Приложение 1}} 
\addcontentsline{toc}{section}{Приложение 1}
\subsection*{Исходный код лексического анализатора}
\lstinputlisting{../src/Lexer.hs}
\clearpage

\section*{\titleline[r]{Приложение 2}} 
\addcontentsline{toc}{section}{Приложение 2}
\subsection*{Исходный код синтаксического анализатора}
\lstinputlisting{../src/Parser.hs}
\clearpage

\section*{\titleline[r]{Приложение 3}} 
\addcontentsline{toc}{section}{Приложение 3}
\subsection*{Исходный код семантического анализатора}
\lstinputlisting{../src/TypeCheck.hs}
\clearpage

\end{document}